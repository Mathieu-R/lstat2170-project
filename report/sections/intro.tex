\section{Introduction}

The aim of this project is to analyse the mean monthly temperature (in $\SI{}{\celsius}$) in the city of Recife (Brazil) from $1986$ to $1995$.

First, we will begin by an analysis of the data. We will first check for any trend, seasonalities or variability of the variance and apply ad-hoc method in order to be able to analyze the autocorrelation and partial-autocorrelation plots (respectively abbreviated as \textit{ACF} and \textit{PACF}) of the data. These plots should give us a first intuition of a possible model to model our data.
We will then use automatic model selection using different criterions like the AIC, BIC in order to confirm our intuition and help us to select two or three final models.
We will finally test our model(s) (analyse of coefficient significance, analysis of the residuals, prediction power) in order to choose the best one and use it to make a forecast for the next $18$ months.